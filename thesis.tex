%
% Niniejszy plik stanowi przykład formatowania pracy magisterskiej na
% Wydziale MIM UW.  Szkielet użytych poleceń można wykorzystywać do
% woli, np. formatując własna prace.
%
% Zawartość merytoryczna stanowi oryginalnosiagniecie
% naukowosciowe Marcina Wolinskiego.  Wszelkie prawa zastrzeżone.
%
% Copyright (c) 2001 by Marcin Woliński <M.Wolinski@gust.org.pl>
% Poprawki spowodowane zmianami przepisów - Marcin Szczuka, 1.10.2004
% Poprawki spowodowane zmianami przepisow i ujednolicenie
% - Seweryn Karłowicz, 05.05.2006
% Dodanie wielu autorów i tłumaczenia na angielski - Kuba Pochrybniak, 29.11.2016

% dodaj opcję [licencjacka] dla pracy licencjackiej
% dodaj opcję [en] dla wersji angielskiej (mogą być obie: [licencjacka,en])
\documentclass{pracamgr}
\usepackage[hidelinks]{hyperref}
\usepackage{tikz}
% \usepackage{refcheck}
\usepackage{dirtree}
\usetikzlibrary{shapes,arrows,positioning}

\tikzstyle{decision} = [diamond, draw, fill=blue!20,
    text width=4.5em, text badly centered, node distance=3cm, inner sep=0pt]
\tikzstyle{block} = [rectangle, draw, fill=gray!20,
    text centered, rounded corners, minimum height=4em, minimum width=4cm]
\tikzstyle{line} = [draw, -latex']
\tikzstyle{cloud} = [draw, ellipse,fill=red!20, node distance=3cm,
    minimum height=2em]

% W tym szablonie początki rozdziałów są na nieparzystych stronach, więc pojawia się dużo pustych.
% W wersji "produkcyjnej" trzeba zakomentować.
% \let\cleardoublepage\clearpage
\DeclareUnicodeCharacter{00A0}{ }

% blokujemy dzielenie wybranych słów
%\hyphenation{InfluxDB}


% Dane magistrantów:
\autor{Dominik Murzynowski}{360180}
\autori{Piotr Zalas}{361374}


\title{Mobilny USOS - aplikacja mobilna dla studentów i pracowników uczelni}
%\tytulang{ Mobile USOS - mobile application for university students and employees}

\kierunek{informatyka}

% informatyka - nie okreslamy zakresu (opcja zakomentowana)
% matematyka - zakres moze pozostac nieokreslony,
% a jesli ma byc okreslony dla pracy mgr,
% to przyjmuje jedna z wartosci:
% {metod matematycznych w finansach}
% {metod matematycznych w ubezpieczeniach}
% {matematyki stosowanej}
% {nauczania matematyki}
% Dla pracy licencjackiej mamy natomiast
% mozliwosc wpisania takiej wartosci zakresu:
% {Jednoczesnych Studiow Ekonomiczno--Matematycznych}

% \zakres{Tu wpisac, jesli trzeba, jedna z opcji podanych wyzej}

% Praca wykonana pod kierunkiem:
% (podać tytuł/stopień imię i nazwisko opiekuna
% Instytut
% ew. Wydział ew. Uczelnia (jeżeli nie MIM UW))
\opiekun{dr. Janiny Mincer-Daszkiewicz\\
  Wydział Matematyki, Informatyki i Mechaniki\\
  }

% miesiąc i~rok:
\date{Czerwiec 2019}

%Podać dziedzinę wg klasyfikacji Socrates-Erasmus:
\dziedzina{
%11.0 Matematyka, Informatyka:\\
%11.1 Matematyka\\
%11.2 Statystyka\\
11.3 Informatyka\\
%11.4 Sztuczna inteligencja\\
%11.5 Nauki aktuarialne\\
%11.9 Inne nauki matematyczne i informatyczne
}

%Klasyfikacja tematyczna wedlug AMS (matematyka) lub ACM (informatyka)
\klasyfikacja{Software and its engineering\\
  Designing software\\
  Software development techniques,\\
  TODO}

% Słowa kluczowe:
\keywords{keczup, USOS, wow, TODO}

% Tu jest dobre miejsce na Twoje własne makra i~środowiska:
\newtheorem{defi}{Definicja}[section]
\usepackage{url}
\usepackage{booktabs}
\usepackage{graphicx}
\usepackage{longtable}
\usepackage{caption}

% koniec definicji
\usepackage{tocloft}
\setlength{\cftchapnumwidth}{1.75em}
\usepackage[acronym, numberedsection]{glossaries}

\makeglossaries

% Nowe obiekty do slownika:
%\newglossaryentry{Luminis}
%{
%    name=Luminis,
%    description={Aplikacja WWW wizualizująca stan lamp ulicznych w czasie rzeczywistym, korzystając z technologii służących do przechowywania szeregów czasowych}
%}


\begin{document}

\maketitle
%tu idzie streszczenie na stronę początkową
\begin{abstract}
  W ramach pracy powstanie aplikacja mobilna dla
  studentów i pracowników. Aplikacja będzie dawać dostęp do najczęściej wykorzystywanych 
  funkcjonalności USOSweb. Nie będzie powielać tego, co wygodnie obsługuje się w serwisie 
  webowym, ale da wygodny dostęp do tych informacji i funkcjonalności, do których student czy 
  pracownik chce sięgać w dowolnym miejscu i czasie, bezpośrednio ze swojego smartfona. 
  Przewidujemy, że w ramach pracy zostaną zaimplementowane m. in. moduły do obsługi 
  sprawdzianów i ocen, zarówno po stronie studenta jak i pracownika, a także funkcjonalności 
  związane z kartą biblioteczną i legitymacją studencką. Ważną usługą zapewnianą przez 
  aplikację będzie powiadamianie o zdarzeniach w czasie rzeczywistym, np. zdarzeniach 
  generowanych w module Wnioski. Aplikacja zostanie wdrożona na UW i docelowo na 
  wszystkich uczelniach korzystających z USOS. Do aplikacji dołączymy dokumentację opisującą 
  jej najważniejsze aspekty. Aplikacja będzie dostępna na platformie Android.
\end{abstract}

\tableofcontents
%\listoffigures
%\listoftables

\chapter{Wstęp}

Wzmianka o modułach wniesionych przez UŁ, oraz że aplikacja jest dostępna
w dwóch wersjach językowych.

% We wstępie Słownik pojęć, Co to właściwie jest, opis firmy, czemu wybraliśmy ten projekt,(?wizja?)
%Przedstawienie problemu
%Przedstawienie produktu
%Opis użytkowników i ich cele
%Funkcje programu
%Licencja
%Wykorzystane technologie
%\addcontentsline{toc}{chapter}{Wstęp}

\chapter{Wizja}

\section{Aplikacje mobilne na innych uczelniach}

Na zlecenie Międzyuniwersyteckiego Centrum Informatyzacji przeprowadzono
badanie na temat wykorzystania aplikacji mobilnych przez uczelnie w
Polsce i na świecie. W wyniku tych badań ustalono, że aplikacje mobilne
cieszą się dużą popularnością wśród studentów i pracowników uczelni, ale
między polskimi a zagranicznymi uczelniami występują duże rozbieżności w
pojmowaniu roli tychże aplikacji.

W Polsce dominuje myślenie, że aplikacja mobilna dla uczelni ma służyć głównie
wymianie informacji pomiędzy studentami, pracownikami i władzami uczelni.
Systemy informatycznie uczelni mają usprawniać formalne procesy takie jak
imatrykulacja, przyjmowanie podań, wystawianie ocen, rejestracje na zajęcia.
Nie ma w nich miejsca na zdjęcia studentów lub pracowników w luźnej koszuli
podczas rejsu żaglówką na mazurach. Zamieszczone treści i komunikaty napisane
są językiem wskazującym na urzędowy charakter relacji pomiędzy użytkownikami
systemu.

W przypadku uczelni zagranicznych aplikacje mobilne mają mniej formalny
charakter. Oprócz pełnienia funkcji łączącej studentów, wykładowców i władze
uczelni pełnią także funkcję społecznościową. Organizacje studenckie tych
uczelni mają swoje wydzielone miejsce w aplikacji mobilnej, gdzie mogą się
promować, informować o ciekawych wydarzeniach i spotkaniach mających miejsce
w obrębie uczelnianego kampusu. Sama szata graficzna tych części aplikacji
jest mniej surowa, a edytorzy ich treści pozwalają sobie przykładowo na
swobodniejszy dobór grafik i użycie nieformalnego języka. Wydaje się, że nie
ma potrzeby przenoszenia tej praktyki na polski grunt.

\section{Aplikacja mobilna dla USOS}

Na bazie wniosków z przeprowadzonego badania zdecydowano, że Uniwersytecki
System Obsługi Studiów także ma mieć aplikację mobilną. Możliwe były dwa
modele obsługi urządzeń mobilnych przez USOS:

\begin{itemize}
	\item USOSweb w wersji na smartfony i tablety.
	\item Natywna aplikacja mobilna pobierająca dane z USOS API.
\end{itemize}

Realizacja pierwszego wariantu wydawała się być bardziej atrakcyjna z finansowego
punktu widzenia, gdyż nie trzeba by było wtedy tworzyć nowej aplikacji, a tylko
zaadaptować już istniejący serwis poprzez uczynienie go responsywnym. Niestety
ze względu na architekturę USOSweb i naleciałości historyczne zadanie to okazało
się być technicznie niewykonalne w realistycznym horyzoncie czasowym. Ponadto
taka realizacja aplikacji mobilnej pozbawia użytkownika wielu benefitów wypływających
z natywnej implementacji, jak chociażby niezależności od awarii serwerów uczelni.

W związku z powyższymi rozważaniami MUCI podjęło decycję o stworzeniu natywnej
aplikacji Mobilny USOS dla platformy Android. Oczywiście Mobilny USOS z założenia
nie ma reimplementować całej funkcjonalności USOSweb, tylko jej najważniejsze
elementy. Aby wyłonić kluczowe dla studenta moduły postanowiono przeprowadzić
ankietę wśród studentów i samorządu.

\section{Wyniki ankiety wśród studentów i samorządu}

Studenci wydziału MIM UW wypełniający ankietę wpisali takie postulaty dotyczące
oczekiwanych przyszłych funkcjonalności Mobilnego USOS:

\begin{itemize}
	\item Przeglądarka ocen i sprawdzianów.
	\item Plan zajęć wraz z powiadamianiem o zbliżających się zajęciach
	      i kolokwiach.
	\item Dostęp do katalogów i wyszukiwarek.
	\item Mapa uniwersytetu z podziałem na poszczególne wydziały, wraz z
	      możliwością nawigowania do wybranej lokalizacji.
	\item Obsługa modułu rejestracji.
	\item Możliwość podpinania przedmiotów.
\end{itemize}

Do powyższej listy samorząd studencki wydziału MIM UW dopisałby też:

\begin{itemize}
	\item Informacje o wydziale przydatne dla studenta -- godziny otwarcia sekcji
	      studenckiej, terminy dyżuru dziekana wraz z elektronicznymi zapisami na
	      dyżur, aktualny plan sesji.
	\item Możliwość wypełniania ankiet wraz z powiadomieniami o nowej ankiecie.
	\item Informacje o decyzjach i statusie podań.
	\item Kalendarz rejestracji wraz z powiadomieniami.
	\item Synchronizację kalendarza widocznego w USOSweb z kalendarzem telefonu.
\end{itemize}

Samorząd wyraził też obawę, że aplikacja mająca za dużo funkcji, niekoniecznie
użytecznych dla większości studentów, będzie mniej przyjazna w obsłudze.

Jako rezultat przeprowadzonych ankiet wytypowano następujące podstawowe moduły,
które przyjęto do realizacji: ankiety, katalog, oceny, plan zajęć, sprawdziany,
przydatne informacje o uczelni.

\section{Oczekiwania pracowników i administracji uczelni}

Logika USOS związana z pracownikami jest bez porównania znacznie bardziej
skomplikowana od logiki związanej z studentami. Nie ma sensu przepisywać do
aplikacji mobilnej modułów związanych np. z uzupełnianiem syllabusów, gdyż
nie dość, że takie moduły są używane stosunkowo rzadko, to jeszcze operacje
z ich użyciem wykonuje się zdecydowanie wygodniej z poziomu komputera. Postanowiliśmy
zaimplementować tylko te moduły, które byłyby przydatne w sytuacji kiedy pracownik
nie ma dostępu do komputera, np. w trakcie zajęć lub kiedy nadgorliwy student
spotka pracownika na korytarzu swojego wydziału i będzie oczekiwał natychmiastowej
informacji na temat swojej oceny.

W związku z powyższym, do realizacji przyjęliśmy moduł protokołów egzaminacyjnych
i część funkcjonalności sprawdzianów związanej z wystawianiem punktów i ocen.
Oprócz tego pracownik będzie miał w czasie rzeczywistym dostęp do ilości wypełnionych
ankiet, które go dotyczą. Dzięki temu będzie mógł skuteczniej wywierać na studentów
presję w sprawie wypełniania ankiet, co z kolei jest pożądane z punktu widzenia
uczelnianej administracji.

Innym, niezwykle ważnym z punktu widzenia biura prasowego uczelni modułem są
aktualności, które pozwalają pracownikom biura publikować artykuły wyświetlane
w Mobilnym USOS. Autor artykułu powinien mieć możliwość wybrania grupy odbiorców,
którym wyświetli się notka prasowa.

Przy projektowaniu aplikacji należy uwzględnić różnice między uczelniami.
Przykładowo, każda uczelnia we własnym zakresie wdraża nowe wersje USOS API.
Ponadto uczelnie mają wdrożone swoje własne standardy identyfikacji wizualnej.
Należy zapewnić uczelniom możliwość personalizacji wybranych elementów szaty
graficznej Mobilnego USOS. Przy wdrażaniu aplikacji należy pamiętać o
sprawdzeniu kompatybilności z zainstalowaną przez uczelnię wersją serwera API.
Z wymienionych wyżej powodów zdecydowano, że każda uczelnia będzie miała własną
wersję Mobilnego USOS tworzoną z wspólnego szablonu.

Największe polskie uczelnie są instytucjami publicznymi i muszą spełniać wytyczne
dostępności narzucane przez polskie prawo. W przypadku aplikacji mobilnych
największy nacisk położony jest na spełnienie norm dotyczących kontrastu i prawidłowe
współdziałanie z oprogramowaniem wspierającym osoby niewidome, takim jak Google
TalkBack.

\chapter{Specyfikacja wymagań}

\section{Wymogi związane z finansowaniem}

Odpowiednie loga na ekranie logowania, sprawy dostępności.

\section{Logowanie do aplikacji}

\section{Dashboard}

\section{Oceny}

\section{Sprawdziany}

\section{Protokoły}

\section{itd...}

\chapter{Technologie}

\section{Android}

\section{Firebase}

\section{Gradle}

\section{Jackson}

\section{Java}

Jest beznadziejna

\section{Kotlin}

Tutaj coś o korutynach, null safety, własnościach funkcyjnych i bogatej bibliotece standardowej

\section{OkHttp}

\section{Retrofit}

\chapter{Szczegóły implementacji}

\section{Ogólna budowa aplikacji}

3 główne activity: SplashScreen, LoginActivity, MainActivity.
Poszczególne ekrany w obrębie MainActivity zrealizowane jako
Fragmenty.

\section{Komunikacja z USOS API}

\section{System obsługujący powiadomienia}

Firebase + powiadomienia

\section{Budowa typowego fragmentu}

Szablony bazowe klas dla dwóch rodzajów fragmentów:
wyświetlających pojedynczy widok oraz widoku listy.

\chapter{Wdrożenie w skali polski}

\section{Wyzwania związane z wdrażaniem}

Konieczność zachowania kompatybilności pomiędzy różnymi
wersjami Mobilnego USOS i USOS API. Dużo uczelni.

\section{System budowania aplikacji}

Tutaj troszkę o private.gradle, kluczach i personalizacji aplikacji,
może o rzeczach, które uczelnia ma dostarczyć.

\section{Fabric}

Wystawianie aplikacji do testów

\section{Crashlytics}

Łapanie błędów z testów

\section{Jenkins}

Automatyzacja budowania paczek w Jenkins

\section{Wdrożenie do sklepu Play}

Jeden wzorzec, cała czarna robota ze względów logistycznych spada na uczelnię

\chapter{Podsumowanie}

Perspektywy na przyszłość - rozwój przez PW, zewnętrznych magistrantów, apetyt rośnie w miarę jedzenia.

\appendix

\newpage
\listoffigures

% \nocite{*}

\begin{thebibliography}{99}
\addcontentsline{toc}{chapter}{Bibliografia}

%\bibitem[Bea65]{beaman} Juliusz Beaman, \textit{Morbidity of the Jolly
%   function}, Mathematica Absurdica, 117 (1965) 338--9.
%\bibitem[Ajax]{AJAXwikipl} AJAX - Wikipedia \url{https://pl.wikipedia.org/wiki/AJAX} (dostęp 21.03.2017)

\end{thebibliography}

\end{document}


%%% Local Variables:
%%% mode: latex
%%% TeX-master: t
%%% coding: latin-2
%%% End:
