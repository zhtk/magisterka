%
% Niniejszy plik stanowi przykład formatowania pracy magisterskiej na
% Wydziale MIM UW.  Szkielet użytych poleceń można wykorzystywać do
% woli, np. formatując własna prace.
%
% Zawartość merytoryczna stanowi oryginalnosiagniecie
% naukowosciowe Marcina Wolinskiego.  Wszelkie prawa zastrzeżone.
%
% Copyright (c) 2001 by Marcin Woliński <M.Wolinski@gust.org.pl>
% Poprawki spowodowane zmianami przepisów - Marcin Szczuka, 1.10.2004
% Poprawki spowodowane zmianami przepisow i ujednolicenie
% - Seweryn Karłowicz, 05.05.2006
% Dodanie wielu autorów i tłumaczenia na angielski - Kuba Pochrybniak, 29.11.2016

% dodaj opcję [licencjacka] dla pracy licencjackiej
% dodaj opcję [en] dla wersji angielskiej (mogą być obie: [licencjacka,en])
\documentclass{pracamgr}
\usepackage[hidelinks]{hyperref}
\usepackage{tikz}
% \usepackage{refcheck}
\usepackage{dirtree}
\usepackage{caption}
\usetikzlibrary{shapes,arrows,positioning}

\tikzstyle{decision} = [diamond, draw, fill=blue!20,
    text width=4.5em, text badly centered, node distance=3cm, inner sep=0pt]
\tikzstyle{block} = [rectangle, draw, fill=gray!20,
    text centered, rounded corners, minimum height=4em, minimum width=4cm]
\tikzstyle{line} = [draw, -latex']
\tikzstyle{cloud} = [draw, ellipse,fill=red!20, node distance=3cm,
    minimum height=2em]

% W tym szablonie początki rozdziałów są na nieparzystych stronach, więc pojawia się dużo pustych.
% W wersji "produkcyjnej" trzeba zakomentować.
% \let\cleardoublepage\clearpage
\DeclareUnicodeCharacter{00A0}{ }

% blokujemy dzielenie wybranych słów
%\hyphenation{InfluxDB}


% Dane magistrantów:
\autor{Dominik Murzynowski}{360180}
\autori{Piotr Zalas}{361374}


\title{Mobilny USOS - aplikacja mobilna dla studentów i pracowników uczelni}
%\tytulang{ Mobile USOS - mobile application for university students and employees}

\kierunek{informatyka}

% informatyka - nie okreslamy zakresu (opcja zakomentowana)
% matematyka - zakres moze pozostac nieokreslony,
% a jesli ma byc okreslony dla pracy mgr,
% to przyjmuje jedna z wartosci:
% {metod matematycznych w finansach}
% {metod matematycznych w ubezpieczeniach}
% {matematyki stosowanej}
% {nauczania matematyki}
% Dla pracy licencjackiej mamy natomiast
% mozliwosc wpisania takiej wartosci zakresu:
% {Jednoczesnych Studiow Ekonomiczno--Matematycznych}

% \zakres{Tu wpisac, jesli trzeba, jedna z opcji podanych wyzej}

% Praca wykonana pod kierunkiem:
% (podać tytuł/stopień imię i nazwisko opiekuna
% Instytut
% ew. Wydział ew. Uczelnia (jeżeli nie MIM UW))
\opiekun{dr. Janiny Mincer-Daszkiewicz\\
  Wydział Matematyki, Informatyki i Mechaniki\\
  }

% miesiąc i~rok:
\date{Czerwiec 2019}

%Podać dziedzinę wg klasyfikacji Socrates-Erasmus:
\dziedzina{
%11.0 Matematyka, Informatyka:\\
%11.1 Matematyka\\
%11.2 Statystyka\\
11.3 Informatyka\\
%11.4 Sztuczna inteligencja\\
%11.5 Nauki aktuarialne\\
%11.9 Inne nauki matematyczne i informatyczne
}

%Klasyfikacja tematyczna wedlug AMS (matematyka) lub ACM (informatyka)
\klasyfikacja{Software and its engineering\\
  Designing software\\
  Software development techniques,\\
  TODO}

% Słowa kluczowe:
\keywords{keczup, USOS, wow, TODO}

% Tu jest dobre miejsce na Twoje własne makra i~środowiska:
\newtheorem{defi}{Definicja}[section]
\usepackage{url}
\usepackage{booktabs}
\usepackage{graphicx}
\usepackage{tabu}
\usepackage{caption}

% koniec definicji
\usepackage{tocloft}
\setlength{\cftchapnumwidth}{1.75em}
\usepackage[acronym, numberedsection]{glossaries}

\makeglossaries

% Nowe obiekty do slownika:
%\newglossaryentry{Luminis}
%{
%    name=Luminis,
%    description={Aplikacja WWW wizualizująca stan lamp ulicznych w czasie rzeczywistym, korzystając z technologii służących do przechowywania szeregów czasowych}
%}


\begin{document}

\maketitle
%tu idzie streszczenie na stronę początkową
\begin{abstract}
  W ramach pracy powstanie aplikacja mobilna dla
  studentów i pracowników. Aplikacja będzie dawać dostęp do najczęściej wykorzystywanych 
  funkcjonalności USOSweb. Nie będzie powielać tego, co wygodnie obsługuje się w serwisie 
  webowym, ale da wygodny dostęp do tych informacji i funkcjonalności, do których student czy 
  pracownik chce sięgać w dowolnym miejscu i czasie, bezpośrednio ze swojego smartfona. 
  Przewidujemy, że w ramach pracy zostaną zaimplementowane m. in. moduły do obsługi 
  sprawdzianów i ocen, zarówno po stronie studenta jak i pracownika, a także funkcjonalności 
  związane z kartą biblioteczną i legitymacją studencką. Ważną usługą zapewnianą przez 
  aplikację będzie powiadamianie o zdarzeniach w czasie rzeczywistym, np. zdarzeniach 
  generowanych w module Wnioski. Aplikacja zostanie wdrożona na UW i docelowo na 
  wszystkich uczelniach korzystających z USOS. Do aplikacji dołączymy dokumentację opisującą 
  jej najważniejsze aspekty. Aplikacja będzie dostępna na platformie Android.
\end{abstract}

\tableofcontents
%\listoffigures
\newpage
\listoftables

\chapter{Wstęp}

Wzmianka o modułach wniesionych przez UŁ, oraz że aplikacja jest dostępna
w dwóch wersjach językowych.

% We wstępie Słownik pojęć, Co to właściwie jest, opis firmy, czemu wybraliśmy ten projekt,(?wizja?)
%Przedstawienie problemu
%Przedstawienie produktu
%Opis użytkowników i ich cele
%Funkcje programu
%Licencja
%Wykorzystane technologie

\chapter{Wizja}

\section{Aplikacje mobilne na innych uczelniach}

Na zlecenie Międzyuniwersyteckiego Centrum Informatyzacji przeprowadzono
badania na temat wykorzystania aplikacji mobilnych przez uczelnie w
Polsce i na świecie. W wyniku tych badań ustalono, że aplikacje mobilne
cieszą się dużą popularnością wśród studentów i pracowników uczelni, ale
między polskimi i zagranicznymi uczelniami występują duże rozbieżności w
pojmowaniu roli tych aplikacji.

W Polsce dominuje myślenie, że aplikacja mobilna dla uczelni ma służyć głównie
wymianie informacji pomiędzy studentami, nauczycielami akademickimi i administracją
uczelni. Systemy informatycznie uczelni mają usprawniać formalne procesy, takie jak
wystawianie ocen, informowanie o wynikach kolokwium czy dostarczanie obowiązującego
planu zajęć. Zamieszczone treści i komunikaty napisane są językiem wskazującym na
urzędowy charakter relacji pomiędzy użytkownikami systemu.

W przypadku uczelni zagranicznych aplikacje mobilne mają mniej formalny
charakter. Oprócz pełnienia funkcji łączącej studentów, nauczycieli akademickich
i administracji pełnią także funkcję społecznościową. Organizacje studenckie tych
uczelni mają swoje wydzielone miejsce w aplikacji mobilnej, gdzie mogą się
promować, informować o ciekawych wydarzeniach i spotkaniach mających miejsce
w obrębie uczelnianego kampusu. Sama szata graficzna tych części aplikacji
jest mniej surowa, a edytorzy ich treści pozwalają sobie przykładowo na
swobodniejszy dobór grafik i użycie nieformalnego języka. Nie taką rolę
przewidujemy dla oficjalnej aplikacji będącej uzupełnieniem USOSweb.

\section{Aplikacja mobilna dla USOS}

Na bazie wniosków z przeprowadzonego badania zdecydowano, że Uniwersytecki
System Obsługi Studiów także ma mieć aplikację mobilną. Możliwe były dwa
modele dostępu do danych zgromadzonych w USOS:

\begin{itemize}
	\item USOSweb w wersji na smartfony i tablety.
	\item Natywna aplikacja mobilna pobierająca dane z USOS poprzez USOS API.
\end{itemize}

Realizacja pierwszego wariantu wydawała się być bardziej atrakcyjna z finansowego
punktu widzenia, gdyż nie trzeba by było wtedy tworzyć nowej aplikacji, a tylko
dostosować USOSweb do potrzeb aplikacji mobilnych. Niestety ze względu na architekturę
USOSweb i naleciałości historyczne zadanie to okazało się być technicznie niewykonalne
w realistycznym horyzoncie czasowym. Ponadto taka realizacja aplikacji mobilnej
pozbawia użytkownika wielu benefitów wypływających z natywnej implementacji.
W nowej aplikacji wiele funkcjonalności USOSweb może być dostarczanych w inny
sposób, który jest dostosowany do specyfiki urządzeń mobilnych.

W związku z przytoczonymi argumentami MUCI podjęło decycję o stworzeniu natywnej
aplikacji Mobilny USOS dla platformy Android. Oczywiście Mobilny USOS z założenia
nie ma reimplementować całej funkcjonalności USOSweb, tylko jej najważniejsze
elementy. Aby wyłonić kluczowe dla studenta moduły, postanowiono przeprowadzić
ogólną ankietę wśród studentów Uniwersytetu Mikołaja Kopernika w Toruniu i Uniwersytetu
Śląskiego, oraz nieco bardziej szczegółową ankietę wśród studentów i samorządu
Wydziału MIM UW.

\section{Wyniki ogólnej ankiety przeprowadzonej na UMK i UŚ}

Na początku ankiety poproszono respondentów o podanie podstawowych informacji o
stopniu studiów i trybie studiowania. Ankietę wypełniło 1216 studentów UMK i
703 studentów UŚ.

\begingroup
\centering
\begin{longtabu} to \textwidth { |X[l]|X[l]|X[l]| }
	\hline
	Odpowiedź & Liczba studentów UMK, którzy zaznaczyli daną odpowiedź & Liczba studentów UŚ, którzy zaznaczyli daną odpowiedź\\
	
	\hline
	\multicolumn{3}{|c|}{\textbf{Wskaż stopień swoich studiów}} \\
	\hline
	
	Pierwszego stopnia & 708 & 366\\
	Drugiego stopnia & 303 & 159\\
	Jednolite magisterskie & 189 & 163\\
	Trzeciego stopnia & 43 & 23\\
	Podyplomowe & 8 & 1\\
		
	\hline
	\multicolumn{3}{|c|}{\textbf{Wskaż tryb swoich studiów}} \\
	\hline

	Stacjonarne & 1142 & 606\\
	Niestacjonarne - wieczorowe & 8 & 14\\
	Niestacjonarne - zaoczne & 73 & 90\\
	
	\hline
\end{longtabu}
\captionof{table}{Stopień i tryb studiowania u ankietowanych studentów UMK i UŚ}\label{tbl:studank}
\endgroup
\medskip

Następnie poproszono ankietowane osoby o wskazanie częstotliwości, z jakimi
korzystają z wymienionych modułów USOSweb.

\begingroup
\centering
\begin{longtabu} to \textwidth { |X[l]|X[l]|X[l]| }
	\hline
	Częstotliwość & Liczba studentów UMK, którzy zaznaczyli daną odpowiedź & Liczba studentów UŚ, którzy zaznaczyli daną odpowiedź\\
	
	\hline
	\multicolumn{3}{|c|}{\textbf{Katalog studentów, pracowników, przedmiotów, kierunków studiów}} \\
	\hline
	
	Codziennie & 309 & 42\\
	Raz w tygodniu & 361 & 187\\
	Raz na miesiąc & 332 & 195\\
	Raz na semestr & 134 & 150\\
	Raz na rok & 17 & 25\\
	Jednorazowo & 51 & 59\\
	Nigdy & 12 & 45\\
	
	\hline
	\multicolumn{3}{|c|}{\textbf{Plan zajęć}} \\
	\hline
	Codziennie & 428 & 83\\
	Raz w tygodniu & 459 & 126\\
	Raz na miesiąc & 133 & 95\\
	Raz na semestr & 84 & 75\\
	Raz na rok & 5 & 9\\
	Jednorazowo & 31 & 62\\
	Nigdy & 74 & 253\\
	
	\hline
	\multicolumn{3}{|c|}{\textbf{USOSmail (formularz poczty elektronicznej)}} \\
	\hline
	Codziennie & 847 & 18\\
	Raz w tygodniu & 205 & 36\\
	Raz na miesiąc & 62 & 57\\
	Raz na semestr & 17 & 47\\
	Raz na rok & 9 & 19\\
	Jednorazowo & 27 & 84\\
	Nigdy & 49 & 442\\	
	
	\hline
	\multicolumn{3}{|c|}{\textbf{Rejestracje na przedmioty, zajęcia, egzaminy}} \\
	\hline
	Codziennie & 17 & 12\\
	Raz w tygodniu & 61 & 21\\
	Raz na miesiąc & 117 & 56\\
	Raz na semestr & 957 & 523\\
	Raz na rok & 19 & 19\\
	Jednorazowo & 34 & 33\\
	Nigdy & 11 & 37\\
	
	\hline
	\multicolumn{3}{|c|}{\textbf{Sprawdziany}} \\
	\hline
	Codziennie & 65 & 19\\
	Raz w tygodniu & 137 & 55\\
	Raz na miesiąc & 159 & 61\\
	Raz na semestr & 167 & 88\\
	Raz na rok & 36 & 6\\
	Jednorazowo & 110 & 46\\
	Nigdy & 538 & 428\\
	
	\hline
	\multicolumn{3}{|c|}{\textbf{Oceny}} \\
	\hline
	Codziennie & 368 & 76\\
	Raz w tygodniu & 350 & 119\\
	Raz na miesiąc & 273 & 164\\
	Raz na semestr & 202 & 261\\
	Raz na rok & 5 & 5\\
	Jednorazowo & 15 & 30\\
	Nigdy & 3 & 48\\
	
	\hline
	\multicolumn{3}{|c|}{\textbf{Podpięcia przedmiotów pod programy studiów}} \\
	\hline
	Codziennie & 10 & 9\\
	Raz w tygodniu & 40 & 17\\
	Raz na miesiąc & 98 & 61\\
	Raz na semestr & 900 & 324\\
	Raz na rok & 27 & 16\\
	Jednorazowo & 88 & 66\\
	Nigdy & 52 & 207\\
	
	\hline
	\multicolumn{3}{|c|}{\textbf{Podania studenckie}} \\
	\hline
	Codziennie & 7 & 6\\
	Raz w tygodniu & 15 & 11\\
	Raz na miesiąc & 90 & 32\\
	Raz na semestr & 323 & 143\\
	Raz na rok & 144 & 38\\
	Jednorazowo & 265 & 115\\
	Nigdy & 372 & 355\\
	
	\hline
	\multicolumn{3}{|c|}{\textbf{Wnioski (o stypendia naukowe i socjalne, akademik, zapomogę itp.)}} \\
	\hline
	Codziennie & 9 & 4\\
	Raz w tygodniu & 12 & 15\\
	Raz na miesiąc & 84 & 51\\
	Raz na semestr & 263 & 157\\
	Raz na rok & 293 & 62\\
	Jednorazowo & 184 & 91\\
	Nigdy & 370 & 319\\
	
	\hline
	\multicolumn{3}{|c|}{\textbf{Międzynarodowa wymiana studencka}} \\
	\hline
	Codziennie & 4 & 4\\
	Raz w tygodniu & 4 & 4\\
	Raz na miesiąc & 24 & 6\\
	Raz na semestr & 66 & 14\\
	Raz na rok & 49 & 14\\
	Jednorazowo & 102 & 43\\
	Nigdy & 966 & 615\\
	
	\hline
	\multicolumn{3}{|c|}{\textbf{Ankiety}} \\
	\hline
	Codziennie & 5 & 6\\
	Raz w tygodniu & 33 & 16\\
	Raz na miesiąc & 105 & 110\\
	Raz na semestr & 567 & 178\\
	Raz na rok & 95 & 58\\
	Jednorazowo & 234 & 215\\
	Nigdy & 177 & 116\\
	
	\hline
	\multicolumn{3}{|c|}{\textbf{Płatności}} \\
	\hline
	Codziennie & 5 & 4\\
	Raz w tygodniu & 19 & 7\\
	Raz na miesiąc & 159 & 50\\
	Raz na semestr & 162 & 119\\
	Raz na rok & 101 & 63\\
	Jednorazowo & 324 & 189\\
	Nigdy & 446 & 269\\
	
	\hline
\end{longtabu}
\captionof{table}{Jak często studenci UMK i UŚ korzystają z poszczególnych modułów USOSweb}\label{tbl:usoswebank}
\endgroup
\medskip

Następnie zapytano o sprawy związane z aplikacją mobilną.

\begingroup
\centering
\begin{longtabu} to \textwidth { |X[l]|X[l]|X[l]| }
	\hline
	Odpowiedź & Liczba studentów UMK, którzy zaznaczyli daną odpowiedź & Liczba studentów UŚ, którzy zaznaczyli daną odpowiedź\\
	
	\hline
	\multicolumn{3}{|p{\dimexpr 3\tabucolX+3\tabcolsep+\arrayrulewidth\relax}|}{\textbf{Czy Twoim zdaniem przydatna byłaby aplikacja na urządzenia mobilne oferująca funkcje serwisu USOSweb?}} \\
	\hline
	
	Tak & 1172 & 590\\
	Nie & 43 & 109\\
	
	\hline
	\multicolumn{3}{|p{\dimexpr 3\tabucolX+3\tabcolsep+\arrayrulewidth\relax}|}{\textbf{Jakie moduły serwisu USOSweb powinny być obsługiwane w takiej aplikacji?}} \\
	\hline
	
	Katalogi & 832 & 394\\
	Plan zajęć & 1127 & 515\\
	USOSmail & 1087 & 181\\
	Rejestracje & 726 & 530\\
	Sprawdziany & 410 & 268\\
	Oceny & 1179 & 624\\
	Podpięcia & 459 & 237\\
	Podania studenckie & 248 & 177\\
	Wnioski & 355 & 259\\
	Międzynarodowa wymiana studencka & 126 & 75\\
	Ankiety & 267 & 94\\
	Płatności & 310 & 220\\
	Inne & 25 & 13\\
	
	\hline
	\multicolumn{3}{|p{\dimexpr 3\tabucolX+3\tabcolsep+\arrayrulewidth\relax}|}{\textbf{Jakie systemy operacyjne posiadasz w swoich urządzeniach mobilnych?}} \\
	\hline
	
	Android & 1028 & 540\\
	iOS & 189 & 90\\
	Windows Phone & 201 & 113\\
	Inne & 22 & 23\\
	Nie wiem & 21 & 34\\
	Nie posiadam urządzenia mobilnego & 15 & 27\\
	
	\hline
\end{longtabu}
\captionof{table}{Pytania o aplikację mobilną u ankietowanych studentów UMK i UŚ}\label{tbl:mobilank}
\endgroup
\medskip

Z przeprowadzonej ankiety wyciągneliśmy następujące wnioski:
\begin{itemize}
	\item TODO % TODO
\end{itemize}

\section{Wyniki ankiety wśród studentów i samorządu Wydziału MIM UW}

Studenci Wydziału MIM UW wypełniający ankietę wskazali następujące moduły, które
chcieliby znaleźć w Mobilnym USOS:

\begin{itemize}
	\item Przeglądarkę ocen i sprawdzianów.
	\item Plan zajęć wraz z powiadamianiem o zbliżających się zajęciach
	      i kolokwiach.
	\item Katalogi wraz z wyszukiwarkami.
	\item Mapę uniwersytetu z podziałem na poszczególne wydziały, wraz z
	      możliwością nawigowania do wybranej lokalizacji.
	\item Rejestracje na zajęcia.
	\item Podpięcia.
\end{itemize}

Do tej listy samorząd studencki Wydziału MIM UW dopisał:

\begin{itemize}
	\item Informacje o wydziale przydatne dla studenta -- godziny otwarcia sekcji
	      studenckiej, terminy dyżuru dziekana wraz z elektronicznymi zapisami na
	      dyżur, aktualny plan sesji.
	\item Wypełnianie ankiet wraz z powiadomieniami o nowych ankietach.
	\item Decyzje i statusy podań.
	\item Kalendarz rejestracji wraz z powiadomieniami.
	\item Synchronizację kalendarza widocznego w USOSweb z kalendarzem telefonu.
\end{itemize}
Samorząd wyraził też obawę, że aplikacja mająca za dużo funkcji, niekoniecznie
użytecznych dla większości studentów, będzie mniej przyjazna w obsłudze.

W wyniku przeprowadzonych ankiet przyjęto do realizacji następujące moduły:
ankiety, katalog, oceny, plan zajęć, sprawdziany, przydatne informacje o uczelni.

\section{Oczekiwania nauczycieli akademickich i administracji uczelni}

Logika USOS związana z pracownikami jest bez porównania znacznie bardziej
skomplikowana od logiki związanej ze studentami. Nie ma sensu przepisywać do
aplikacji mobilnej modułów związanych np. z uzupełnianiem sylabusów, gdyż
nie dość, że takie moduły są używane stosunkowo rzadko, to jeszcze operacje
z ich użyciem wykonuje się zdecydowanie wygodniej z poziomu komputera. Postanowiliśmy
zaimplementować tylko te moduły, które byłyby przydatne w sytuacji kiedy pracownik
nie ma dostępu do komputera, np. w trakcie zajęć lub kiedy nadgorliwy student
spotka pracownika na korytarzu swojego wydziału i będzie oczekiwał natychmiastowej
informacji na temat swojej oceny.

W związku z powyższym, do realizacji przyjęliśmy moduł ankiet, protokołów
egzaminacyjnych i część funkcjonalności sprawdzianów związanej z wystawianiem
punktów i ocen. Kolejnym modułem, niezwykle ważnym z punktu widzenia biura prasowego
uczelni, są aktualności, które pozwalają pracownikom biura publikować artykuły
wyświetlane w Mobilnym USOS.

Przy projektowaniu aplikacji należy uwzględnić różnice między uczelniami.
Przykładowo, każda uczelnia we własnym zakresie wdraża nowe wersje USOS API.
Ponadto uczelnie mają wdrożone swoje własne standardy identyfikacji wizualnej.
Należy zapewnić uczelniom możliwość personalizacji wybranych elementów szaty
graficznej Mobilnego USOS. Przy wdrażaniu aplikacji należy pamiętać o
sprawdzeniu kompatybilności z zainstalowaną przez uczelnię wersją serwera USOS API.
Z wymienionych powodów zdecydowano, że każda uczelnia będzie miała własną
wersję Mobilnego USOS tworzoną ze wspólnego szablonu.

Największe polskie uczelnie są instytucjami publicznymi i muszą spełniać wytyczne
dostępności narzucane przez polskie prawo. W przypadku aplikacji mobilnych
największy nacisk położony jest na spełnienie norm dotyczących kontrastu i prawidłowe
współdziałanie z oprogramowaniem wspierającym osoby niewidome, takim jak Google
TalkBack.

\chapter{Specyfikacja wymagań}

\section{Wymogi związane z finansowaniem}

Odpowiednie loga na ekranie logowania, sprawy dostępności, dwa języki (?).

\section{Logowanie do aplikacji}

\section{Dashboard}

\section{Oceny}

\section{Sprawdziany}

\section{Protokoły}

\section{Moje legitymacje i Moje eID}

\section{itd...}

\chapter{Technologie}

W tym rozdziale przedstawiamy użyte technologie, biblioteki i narzędzia.

\section{Android}

Mobilny USOS jest napisany dla systemu operacyjnego Android \cite{android}.
Każda aplikacja działająca na tej platformie musi być napisana z użyciem Android
SDK. Do rozwoju aplikacji wykorzystaliśmy środowisko programistyczne Android Studio
\cite{androidstudio}, które jest oficjalnie zalecane przez Google. Android Studio
posiada odpowiednie wtyczki, które automatyzują obsługę emulatorów urządzeń, proces
wgrywania i odpluskwiania aplikacji na urządzeniu oraz czynności wykonywane podczas
instalacji Android SDK. Dużym wyzwaniem przy pisaniu aplikacji Androidowych jest
zachowanie kompatybilności pomiędzy różnymi wersjami systemu operacyjnego. Pomaga
w tym biblioteka \textit{Android Support Library} \cite{androidsupportlibrary},
która zapewnia odpowiednią warstwę abstrakcji nad wywołaniami specyficznymi dla
poszczególnych wersji systemu.

\section{Crashlytics}

Crashlytics \cite{crashlytics} jest usługą, która pozwala monitorować błędy
występujące w aplikacji. Składa się ona z biblioteki dołączanej do aplikacji
oraz strony internetowej, na której można przejrzeć częstotliwość występowania
błędów wraz z ich dokładnymi opisami. W sytuacji gdy pojawia się nowy błąd albo
w nagły sposób wzrasta częstotliwość występowania danego błędu, Crashlytics wysyła
do wszystkich programistów emaila z ostrzeżeniem. Raportowanie błędów na poziomie
aplikacji działa na dwa sposoby. Programista może zarejestrować globalną funkcję,
która łapie wszystkie niezłapane wyjątki i wysyła je do Crashlytics. Ponadto przy
pomocy dołączonej biblioteki możliwe jest ręcznie wysłanie informacji o błędzie.

\section{Fabric}

Fabric \cite{fabric} łączy funkcjonalność Crashlytics z prostą analityką opartą na
danych przesyłanych przez aplikację oraz możliwością tworzenia zamkniętych
beta-testów. W Fabric można między innymi sprawdzić, ilu użytkowników korzystało
z aplikacji w danym przedziale czasowym i regionie, ile trwała jedna sesja oraz
ile tych sesji było. Wszystkie dane są zanonimizowane i wyświetlane w sposób
zagregowany. W związku z planowanym wyłączeniem serwisu Fabric większość jego
usług została przeniesiona do Firebase.

\section{Firebase}

Firebase \cite{firebase} dostarcza pakiet usług do tworzenia bezserwerowych
(ang. \textit{serverless}) aplikacji. Jest on znany głównie z usługi składowania
danych w chmurze, z której jednak nie korzystamy. Kluczową usługą Firebase
wykorzystywaną przez Mobilnego USOS jest \textit{Firebase Cloud Messaging}
\cite{firebasecm}, która służy do wysyłania powiadomień z USOS API do określonego
urządzenia z zainstalowaną aplikacją. Wykorzystujemy tę usługę do odbierania
powiadomień o zmianie oceny lub punktów w sprawdzianach. Po wyłączeniu Fabric
Firebase będzie także służył do zbierania informacji o awariach Mobilnego USOS.

\section{Git, Gerrit i Jenkins}

Do składowania kodu wykorzystujemy system kontroli wersji Git \cite{git}. Każdy
napisany przez nas kawałek kodu trafia do systemu Gerrit \cite{gerrit}, w którym
następuje proces opiniowania kodu. Gerrit jest połączony z serwerem Jenkins
\cite{jenkins}, który wykonuje podstawowe testy przesłanego kodu.

\section{Gradle}

Gradle \cite{gradle} jest głównym systemem budowania aplikacji napisanych w
językach Java i Kotlin wykorzystywanym przez platformę Android. Skrypty Gradle
są napisane w języku Groovy. Dzięki odrzuceniu formatu XML wykorzystywanego przez
narzędzia takie jak Ant i Maven na rzecz pełnoprawnego języka programowania
uzyskano dużą elastyczność i zwięzłość wymaganą w naszym projekcie. Z naszego
punktu widzenia kluczowa jest możliwość definiowania wielu odmian aplikacji, po
jednej dla każdej uczelni. Warianty są tworzone automatycznie na podstawie osobnego,
tajnego pliku zawierającego listę uczelni i klucze dostępowe do uczelnianego serwera
USOS API. Ważne dla nas jest to, że możemy uczelni wysłać tylko część pliku zawierającego
jej dane dostępowe i ta część może być bezproblemowo użyta do zbudowania aplikacji.

\section{Jackson}

Jackson \cite{jackson} jest biblioteką do deserializacji danych zapisanych w
formacie JSON. Jest ona o tyle kluczowa, że praktycznie wszystkie dane przesyłane
przez USOS API są zapisane w tym formacie. W porówaniu do konkurencyjnej biblioteki
Gson \cite{gson}, Jackson zapewnia większą kontrolę nad procesem deserializacji
oraz posiada rozszerzenia do obsługi języka Kotlin.

\section{Java}

Java \cite{java} jest natywnym językiem, w którym się pisze aplikacje dla platformy
Android. Niestety sposób obsługi tego języka przez Androida jest dysfunkcyjny.
W naszych zastosowaniach większość interesującej funkcjonalności biblioteki
standardowej jest niedostępna, gdyż jest ona podłączana do obrazu aplikacji podczas
ładowania Mobilnego USOS przez system operacyjny. Im starsza wersja systemu
zainstalowanego na telefonie, tym mniejsza część biblioteki standardowej jest
dostępna. W związku z tym, że Mobilny USOS wspiera bardzo stare wersje systemu
praktycznie wszystkie interesujące pakiety wprowadzone w Java 8 i nowsze są dla
nas niedostępne.

\section{Kotlin}

Odpowiedzią na problemy związane z językiem Java jest język Kotlin \cite{kotlin}.
System typów w języku Kotlin wprowadza rozróżnienie zmiennych na takie, które mogą
przyjmować wartość null i takie, które nie mogą. Specyfikacja języka Kotlin zawiera
dość zaawansowaną inferencję typów, dzięki której kompilator może wyłapać typowe błędy
związane z obsługą wartości null. Ponadto w języku zdefiniowane są specjalne
operatory, które pozwalają w zwięzły i czytelny sposób obsługiwać zmienne mogące
przyjmować wartość null. Język Kotlin jest mocno wzorowany na językach funkcyjnych,
co znalazło odzwierciedlenie w bibliotece standardowej, która udostępnia typowo
funkcyjne operacje na kolekcjach, takie jak map i filter. Zmienne i klasy
kolekcji z biblioteki standardowej, takie jak listy i mapy, są domyślnie niezmienialne
(ang. \textit{immutable}). Kolejnym bardzo przydatnym mechanizmem wprowadzonym w
języku Kotlin są korutyny. Wykorzystujemy je do czytelniejszego zapisu wielowątkowego
kodu. Dzięki mechanizmowi kontynuacji unikamy zjawiska znanego jako \textit{callback hell}
przy obsłudze wywołań USOS API, które było dla nas źródłem wielu problemów w
początkowej fazie projektu.

Wszystkie wymienione własności sprawiły, że Kotlin został uznany przez
Google za oficjalnie wspierany język, w którym można programować aplikacje
Androidowe \cite{kotlinandroid}. Znalazło to swoje odzwierciedlenie w Android SDK,
które jest dostosowane do tego języka.

\section{OkHttp}

OkHttp \cite{okhttp} jest biblioteką do asynchronicznej obsługi połączeń SSL.
Podczas inicjacji tworzy ona sobie wewnętrzną pulę wątków, które są wykorzystywane
podczas wykonywania połączeń. Programista może modyfikować parametry połączenia,
na przykład wstrzykiwać wartości nagłówków lub podmieniać dane pobierane przez
strumień. Biblioteka ta pełni u nas funkcję pomocniczą, jest ona wykorzystywana
przez bibliotekę Retrofit do pobierania danych z USOS API.

\section{Picasso}

Picasso \cite{picasso} jest pomocniczą biblioteką, która zajmuje się obsługą
obrazków zgromadzonych na zewnętrznych serwerach. Dzięki niej możemy pobrać
zdjęcie zapisane na serwerze USOS API i automatycznie zapisać je w pamięci
tymczasowej urządzenia. Dzięki temu unikamy przeciążania serwerów, a samo zdjęcie
jest dostępne także po utracie połączenia sieciowego. Po pobraniu zdjęcia
jest ono automatycznie konwertowane do Androidowej mapy bitowej i wyświetlane w
widoku wskazanym przy uruchamianiu pobierania. Cała operacja odbywa się asynchronicznie.

\section{Retrofit}

Retrofit \cite{retrofit} jest główną biblioteką wykorzystywaną przez nas do
komunikacji z USOS API. Generuje ona klasy odpowiedzialne za pobieranie i
deserializację przysłanych danych z formatu JSON do Javowych klas. Potrzebne klasy
są generowane przy pomocy mechanizmu refleksji na podstawie prostego interfejsu
zawierającego odpowiednio zaanotowane sygnatury funkcji.

\section{Usługi Google dla Androida}

Mobilny USOS w znacznym stopniu wykorzystuje usługi Google. Sklep Google Play
\cite{googleplay} służy do dystrybucji gotowej aplikacji wśród użytkowników
końcowych. Mapy Google \cite{googlemaps} są osadzane wewnątrz aplikacji i służą
do pokazywania lokalizacji jednostek. Większość akcji związanych z kalendarzem
odsyła użytkownika do Kalendarza Google \cite{googlecalendar}, a akcje związane
z wysyłaniem wiadomości email odsyłają do klienta pocztowego, którym zazwyczaj
jest Gmail \cite{gmail}. Decyzja o wykorzystaniu usług Google do realizacji
wymienionych akcji jest o tyle naturalna, że te usługi są wplecione w ekosystem
systemu Android.

\section{ZXing}

ZXing jest akronimem od angielskich słów ,,\textit{Zebra Crossing}'' \cite{zxing}.
Jest to biblioteka wykorzystywana do odczytywania i generowania kodów kreskowych.
Z naszego punktu widzenia istotne jest to, że możemy przy jej pomocy generować
kody w formacie CODE128 wykorzystywanym przez USOS podczas drukowania legitymacji
studenckich. Oprócz tego generujemy też przy pomocy tej biblioteki kody kreskowe
w formacie Aztec.

\chapter{Szczegóły implementacji}

\section{Ogólna budowa aplikacji}

3 główne activity: SplashScreen, LoginActivity, MainActivity.
Poszczególne ekrany w obrębie MainActivity zrealizowane jako
Fragmenty.

\section{Komunikacja z USOS API}

\section{System obsługujący powiadomienia}

Firebase + powiadomienia

\section{Budowa typowego fragmentu}

Szablony bazowe klas dla dwóch rodzajów fragmentów:
wyświetlających pojedynczy widok oraz widoku listy.

\chapter{Wdrożenie w skali polski}

\section{Wyzwania związane z wdrażaniem}

Konieczność zachowania kompatybilności pomiędzy różnymi
wersjami Mobilnego USOS i USOS API. Dużo uczelni.

\section{System budowania aplikacji}

Tutaj troszkę o private.gradle, kluczach i personalizacji aplikacji,
może o rzeczach, które uczelnia ma dostarczyć.

\section{Fabric}

Wystawianie aplikacji do testów

\section{Crashlytics}

Łapanie błędów z testów

\section{Jenkins}

Automatyzacja budowania paczek w Jenkins

\section{Wdrożenie do sklepu Play}

Jeden wzorzec, cała czarna robota ze względów logistycznych spada na uczelnię

\chapter{Podsumowanie}

Perspektywy na przyszłość - rozwój przez PW, zewnętrznych magistrantów, apetyt rośnie w miarę jedzenia.

\appendix

\newpage
\listoffigures

% \nocite{*}

\begin{thebibliography}{99}
\addcontentsline{toc}{chapter}{Bibliografia}

%\bibitem[Bea65]{beaman} Juliusz Beaman, \textit{Morbidity of the Jolly
%   function}, Mathematica Absurdica, 117 (1965) 338--9.
%\bibitem[Ajax]{AJAXwikipl} AJAX - Wikipedia \url{https://pl.wikipedia.org/wiki/AJAX} (dostęp 21.03.2017)

\bibitem[1]{android} Google Inc., \textit{Android -- Documentation for app developers},
	 \url{https://developer.android.com/docs}. [Dostęp: 2019-05-04].

\bibitem[2]{androidstudio} Google Inc., \textit{Android Studio},
	\url{https://developer.android.com/studio}. [Dostęp: 2019-05-07].

\bibitem[3]{androidsupportlibrary} Google Inc., \textit{Android -- Support Library},
	\url{https://developer.android.com/topic/libraries/support-library}. [Dostęp: 2019-05-07].

\bibitem[4]{crashlytics} Google Inc., \textit{Crashlytics}, \url{https://try.crashlytics.com/}. [Dostęp: 2019-05-07].

\bibitem[5]{fabric} Google Inc., \textit{Fabric}, \url{https://get.fabric.io/}. [Dostęp: 2019-05-07].

\bibitem[6]{firebase} Google Inc., \textit{Firebase}, \url{https://firebase.google.com/}. [Dostęp: 2019-05-07].

\bibitem[6a]{firebasecm} Google Inc., \textit{Firebase Cloud Messaging},
	\url{https://firebase.google.com/docs/cloud-messaging}. [Dostęp: 2019-05-09].

\bibitem[7]{git} Software Freedom Conservancy, \textit{Git},
	\url{https://git-scm.com/}. [Dostęp: 2019-05-07].

\bibitem[8]{gerrit} Google Inc., \textit{Gerrit Code Review},
	\url{https://www.gerritcodereview.com/}. [Dostęp: 2019-05-07].

\bibitem[9]{jenkins} Continuous Delivery Foundation, \textit{Jenkins},
	\url{https://jenkins.io/}. [Dostęp: 2019-05-07].

\bibitem[10]{gradle} Gradle Inc., \textit{Gradle},	\url{https://gradle.org/}. [Dostęp: 2019-05-07].

\bibitem[11]{jackson} FasterXML, LLC., \textit{Jackson},
	\url{https://github.com/FasterXML/jackson}. [Dostęp: 2019-05-07].

\bibitem[11a]{gson} Google Inc., \textit{Gson},	\url{https://github.com/google/gson}. [Dostęp: 2019-05-09].

\bibitem[12]{java} Oracle Corporation, \textit{Java}, \url{https://www.java.com/}. [Dostęp: 2019-05-07].

\bibitem[13]{kotlin} Kotlin Foundation, \textit{Kotlin}, \url{https://kotlinlang.org/}. [Dostęp: 2019-05-07].

\bibitem[13a]{kotlinandroid} Maxim Shafirov, \textit{Kotlin on Android. Now official},
	\url{https://blog.jetbrains.com/kotlin/2017/05/kotlin-on-android-now-official/}. [Dostęp: 2019-05-09].

\bibitem[14]{okhttp} Square, Inc., \textit{OkHttp},	\url{https://square.github.io/okhttp/}. [Dostęp: 2019-05-07].

\bibitem[15]{picasso} Square, Inc., \textit{Picasso}, \url{https://square.github.io/picasso/}. [Dostęp: 2019-05-07].

\bibitem[16]{retrofit} Square, Inc., \textit{Retrofit}, \url{https://square.github.io/retrofit/}. [Dostęp: 2019-05-07].

\bibitem[17]{googleplay} Google Inc., \textit{Google Play},	\url{https://play.google.com/store}. [Dostęp: 2019-05-07].

\bibitem[18]{googlemaps} Google Inc., \textit{Google Maps},	\url{https://www.google.com/maps}. [Dostęp: 2019-05-07].

\bibitem[19]{googlecalendar} Google Inc., \textit{Google Calendar},	\url{https://www.google.com/calendar}. [Dostęp: 2019-05-07].

\bibitem[20]{gmail} Google Inc., \textit{Gmail}, \url{https://www.google.com/intl/pl/gmail/about/}. [Dostęp: 2019-05-07].

\bibitem[21]{zxing} Google Inc., \textit{ZXing}, \url{https://github.com/zxing/zxing}. [Dostęp: 2019-05-07].

\end{thebibliography}

\end{document}


%%% Local Variables:
%%% mode: latex
%%% TeX-master: t
%%% coding: latin-2
%%% End:
